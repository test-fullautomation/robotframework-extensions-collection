% --------------------------------------------------------------------------------------------------------------
%
% Copyright 2020-2022 Robert Bosch GmbH

% Licensed under the Apache License, Version 2.0 (the "License");
% you may not use this file except in compliance with the License.
% You may obtain a copy of the License at

% http://www.apache.org/licenses/LICENSE-2.0

% Unless required by applicable law or agreed to in writing, software
% distributed under the License is distributed on an "AS IS" BASIS,
% WITHOUT WARRANTIES OR CONDITIONS OF ANY KIND, either express or implied.
% See the License for the specific language governing permissions and
% limitations under the License.
%
% --------------------------------------------------------------------------------------------------------------

\section{Keywords}

The \rlog{Collection} module of the \pkg\ is the interface between the \textbf{PythonExtensionsCollection} and the
\rfw\ and contains the keyword definitions that can be imported in the following way:

\begin{robotcode}
Library    RobotframeworkExtensions.Collection    WITH NAME    rf.extensions
\end{robotcode}

We recommend to use \rcode{WITH NAME} to shorten the long library name a little bit. That will make the robot code
easier to read.

\subsection{pretty\_print}

The \rcode{pretty_print} keyword logs the content of parameters of any Python data type. Simple data types are logged directly.
Composite data types are resolved before.

\vspace{1ex}

The output contains for every parameter:

\begin{itemize}
   \item the type
   \item the total number of elements inside (e.g. the number of keys inside a dictionary)
   \item the counter number of the current element
   \item the value
\end{itemize}

The trace level for output is \rcode{INFO}. The output is also returned as list of strings.

\newpage

\textbf{Example}

The following \rfw\ code defines - step by step - a parameter of composite data type (nested arrays and dictionaries)
and prints the content of this with \rcode{pretty_print}:

\begin{robotcode}
set_test_variable    @{aItems1}    ${33}
...                                XYZ

set_test_variable    @{aItems}     A
...                                ${22}
...                                ${True}
...                                ${aItems1}

set_test_variable    &{dItems1}    A=${1}
...                                B=${2}

set_test_variable    &{dItems}     K1=value
...                                K2=${aItems}
...                                K3=${10}
...                                K4=${dItems1}

rf.extensions.pretty_print    ${dItems}
\end{robotcode}

\textbf{Result}

\begin{robotlog}
[DOTDICT] (4/1) > {K1} [STR]  :  'value'
[DOTDICT] (4/2) > {K2} [LIST] (4/1) > [STR]  :  'A'
[DOTDICT] (4/2) > {K2} [LIST] (4/2) > [INT]  :  22
[DOTDICT] (4/2) > {K2} [LIST] (4/3) > [BOOL]  :  True
[DOTDICT] (4/2) > {K2} [LIST] (4/4) > [LIST] (2/1) > [INT]  :  33
[DOTDICT] (4/2) > {K2} [LIST] (4/4) > [LIST] (2/2) > [STR]  :  'XYZ'
[DOTDICT] (4/3) > {K3} [INT]  :  10
[DOTDICT] (4/4) > {K4} [DOTDICT] (2/1) > {A} [INT]  :  1
[DOTDICT] (4/4) > {K4} [DOTDICT] (2/2) > {B} [INT]  :  2
\end{robotlog}

Every line of output has to be interpreted strictly from left to right.

For example the meaning of the fifth line of output

\begin{robotlog}
[DOTDICT] (4/2) > {K2} [LIST] (4/4) > [LIST] (2/1) > [INT]  :  33
\end{robotlog}

is:

\begin{itemize}
   \item The type of input parameter \rlog{dItems} is \rlog{dotdict}
   \item The dictionary contains 4 keys
   \item The current line gives information about the second key of the dictionary
   \item The name of the second key is \rlog{K2}
   \item The value of the second key is of type \rlog{list}
   \item The list contains 4 elements
   \item The current line gives information about the fourth element of the list
   \item The fourth element of the list is of type \rlog{list}
   \item The list contains 2 elements
   \item The current line gives information about the first element of the list
   \item The first element of the list is of type \rlog{int} and has the value \rlog{33}
\end{itemize}

Types are encapsulated in square brackets, counter in round brackets and key names are encapsulated in curly brackets.

\newpage

\textbf{Prefix strings}

Prefix strings (\rcode{sPrefix}) can be used to give the lines of output a meaning, or they are used just to print also the name
of the pretty printed variable (\rcode{oData}).

\textbf{Example}

\begin{robotcode}
rf.extensions.pretty_print    ${dItems}    PrefixString
\end{robotcode}

\textbf{Result}

\begin{robotlog}
PrefixString : [DOTDICT] (4/1) > {K1} [STR]  :  'value'
PrefixString : [DOTDICT] (4/2) > {K2} [LIST] (4/1) > [STR]  :  'A'
PrefixString : [DOTDICT] (4/2) > {K2} [LIST] (4/2) > [INT]  :  22
PrefixString : [DOTDICT] (4/2) > {K2} [LIST] (4/3) > [BOOL]  :  True
PrefixString : [DOTDICT] (4/2) > {K2} [LIST] (4/4) > [LIST] (2/1) > [INT]  :  33
PrefixString : [DOTDICT] (4/2) > {K2} [LIST] (4/4) > [LIST] (2/2) > [STR]  :  'XYZ'
PrefixString : [DOTDICT] (4/3) > {K3} [INT]  :  10
PrefixString : [DOTDICT] (4/4) > {K4} [DOTDICT] (2/1) > {A} [INT]  :  1
PrefixString : [DOTDICT] (4/4) > {K4} [DOTDICT] (2/2) > {B} [INT]  :  2
\end{robotlog}

\newpage

\subsection{normalize\_path}

The \rcode{normalize_path} keyword normalizes local paths, paths to local network resources and internet addresses.

\vspace{1ex}

\textbf{Background}

It's not easy to handle paths - and especially the path separators - independent from the operating system.

Under Linux it is obvious that single slashes are used as separator within paths. Whereas the Windows explorer
uses single backslashes. In both operating systems web addresses contains single slashes as separator
when displayed in web browsers.

Using single backslashes within code - as content of string variables - is dangerous because the combination
of a backslash and a letter can be interpreted as escape sequence - and this is maybe not the effect a user wants to have.

To avoid unwanted escape sequences backslashes have to be masked (by the usage of two of them: \rcode{"\\\\"}).
But also this could not be the best solution because there are also applications (like the Windows explorer) that are not able to handle
masked backslashes. They expect to get single backslashes within a path.

Preparing a path for best usage within code also includes collapsing redundant separators and up-level references.
Python already provides functions to do this, but the outcome (path contains slashes or backslashes) depends on the
operating system. And like already mentioned above also under Windows backslashes might not be the preferred choice.

It also has to be considered that redundant separators at the beginning of an address of a local network resource
(like \rcode{\\\\server.com}) and or inside an internet address (like \rcode{https:\\\\server.com}) must \textbf{not} be collapsed!
Unfortunately the Python function \rcode{normpath} does not consider this context.

To give the user full control about the format of a path, independent from the operating system and independent if it's
a local path, a path to a local network resource or an internet address, the keyword \rcode{normalize_path} provides
lot's of parameters to influence the result.

\vspace{1ex}

\textbf{Example 1}

\vspace{1ex}

Variable containing a path with:

\begin{itemize}
   \item different types of path separators
   \item redundant path separators (\textit{but backslashes have to be masked in the definition of the variable, this is \textbf{not} an unwanted redundancy})
   \item up-level references
\end{itemize}

\begin{robotcode}
set_test_variable    ${sPath}    C:\\\\subfolder1///../subfolder2\\\\\\\\../subfolder3\\\\
\end{robotcode}

Printing the content of \rcode{sPath} shows how the path looks like when the masking of the backslashes is resolved:

\begin{robotlog}
C:\\subfolder1///../subfolder2\\\\../subfolder3\\
\end{robotlog}

Usage of the \rcode{normalize_path} keyword:

\begin{robotcode}
${sPath}    rf.extensions.normalize_path    ${sPath}
\end{robotcode}

Result (content of \rcode{sPath}):

\begin{robotlog}
C:/subfolder3
\end{robotlog}

In case we need the Windows version (with masked backslashes instead of slashes):

\begin{robotcode}
${sPath}    rf.extensions.normalize_path    ${sPath}    bWin=${True}
\end{robotcode}

Result (content of \rcode{sPath}):

\begin{robotlog}
C:\\subfolder3
\end{robotlog}

The masking of backslashes can be deactivated:

\begin{robotcode}
${sPath}    rf.extensions.normalize_path    ${sPath}    bWin=${True}    bMask=${False}
\end{robotcode}

Result (content of \rcode{sPath}):

\begin{robotlog}
C:\subfolder3
\end{robotlog}

\vspace{1ex}

\textbf{Example 2}

\vspace{1ex}

Variable containing a path of a local network resource (path starts with two masked backslashes):

\begin{robotcode}
set_test_variable    ${sPath}    \\\\anyserver.com\\\\part1//part2\\\\part3/part4
\end{robotcode}

Result of normalization:

\begin{robotlog}
//anyserver.com/part1/part2/part3/part4
\end{robotlog}

\vspace{1ex}

\textbf{Example 3}

\vspace{1ex}

Variable containing an internet address:

\begin{robotcode}
set_test_variable    ${sPath}    http:\\\\anyserver.com\\\\part1//part2\\\\part3/part4
\end{robotcode}

Result of normalization:

\begin{robotlog}
http://anyserver.com/part1/part2/part3/part4
\end{robotlog}

